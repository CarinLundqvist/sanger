\begin{song}{Fritiof Anderssons Paradmarsch}{fritiofmarsch}
\begin{vers}
Här kommer Fritiof Andersson, det snöar på hans hatt,\\
han går med sång, han går med spel!\\
Hej, mina lustiga bröder!\\
Det knarrar under klackarna, det är vinternatt.\\
Hej, om du vill, säg bara till,\\
så går vi hem till Söder!\\
O, bugen Er I bylingar i bucklor och batong\\
och ställen Er på sidorna, för gränden den är trång.\\
Där går en här som frös och svalt men segrade ändå,\\
den går med sång, den går med spel till Spanien och Bordeaux.\\
\end{vers}
\begin{vers}
Sultanen av Arabiens land, vid Röda flodens krök,\\
ja tänk vad han blir glad ibland!\\
Hej, mina lustiga bröder!\\
Han eldar under oxarna, han väntar vårt besök.\\
Hej, om du vill, säg bara till,\\
så går vi hem till Söder!\\
O, bugen beduiner i burnus och baldakin\\
och ställ er här på sidorna och bjuden oss på vin.\\
Där går en här som frös och svalt men segrade ändå,\\
den går med sång, den går med spel till Spanien och Bordeaux.\\
\end{vers}

\newp

\begin{vers}
Där dansar konung Farao uti Egyptens land,\\
ja tänk vad han blir glad ibland!\\
Hej, mina lustiga bröder!\\
Han reser upp ett sidentält uppå Saharas sand.\\
Hej, om du vill, säg bara till,\\
så går vi hem till Söder!\\
O, bugen er slavinnor uti slöjor och salopp\\
och ställ er här på sidorna och skåden vår galopp!\\
Där går en här som frös och svalt men segrade ändå,\\
den går med sång, den går med spel till Spanien och Bordeaux.\\
\end{vers}
\begin{vers}
I Cadiz och Kastilien där stannar vi en tid,\\
sen går vi några mil igen!\\
Hej, mina lustiga bröder!\\
Då kommer kung Alfonsius och hälsar från Madrid\\
Hej, om du vill, säg bara till,\\
så går vi hem till Söder!\\
O, bugen barcelonere i barett och bardisan\\
och ställ er här på sidorna - släpp fram vår karavan.\\
Där går en här som frös och svalt men segrade ändå,\\
den går med sång, den går med spel till Spanien och Bordeaux.\\
\end{vers}

\newp

\begin{vers}
Där går en här, där går en hop, en liten, men en god\\
den går i ur, den går i skur!\\
Hej, mina lustiga bröder!\\
Den kräver vin och kyssar och den kräver drakars blod!\\
Hej, om du vill, säg bara till,\\
så går vi hem till Söder!\\
O, bugen er I borgare i Birka och Borås\\
och ställ er här på sidorna. Trumpet och valthorn, blås!\\
Där går en här som frös och svalt men segrade ändå,\\
den går med sång, den går med spel till Spanien och Bordeaux.\\
\end{vers}
\begin{vers}
På vägarna vi vandra och på böljorna vi gå,\\
vi gå med spel, vi gå med sång!\\
Hej mina lustiga bröder!\\
I alla sorters väder som vår Herre hittar på!\\
Hej, om du vill, säg bara till,\\
så går vi hem till Söder!\\
O, buga dig du brusande bölja där vi gå!\\
Vårt skepp är själva Friheten, besättningen är blå!\\
Den seglade och frös och svalt men segrade ändå,\\
den går med sång, den går med spel till Spanien och Bordeaux.\\
\end{vers}
\end{song}
