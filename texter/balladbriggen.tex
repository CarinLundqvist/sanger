\begin{song}{Balladen om briggen Blue Bird av Hull}{balladbriggen}
\begin{vers}
Det var Blue Bird av Hull\\
Det var Blue Bird en brigg\\
Som med sviktande stumpar stod på\\
Över soten i snöstorm med nerisad rigg\\
Själva julafton sjuttiotvå\\
- Surra svensken till rors, han kan dreja en spak\\
Ropa skepparn\\
- Allright boys, lös av!\\
Och Karl Stranne från Smögen\\
Blev surrad till rors\\
På Blue Bird som var dömd att bli vrak\\
\end{vers}
\begin{vers}
Han fick Hållö-fyrs blänk\\
Fast av snöglopp och stänk\\
Han stod halvblind\\
Han fick den i lov\\
Och i lä där låg Smögen\\
Hans hem där hans mor\\
Just fått brevet från Middelsborough\\
- Nå vad säger du Karl?\\
- Går hon klar?\\
- Nej, kapten!\\
- Vi får blossa för här är det slut\\
- Vi har Hållö om styrbord och brott strax i lä\\
- Ut med ankarna båtarna ut\\
Men hon red inte upp\\
Och hon fick ett par brott\\
Som tog båten dom hade gjort klart\\
- Jag tror nog, sa Karl Stranne Att far min gått ut\\
- Emot oss, jag litar på far!\\
\end{vers}
\begin{vers}
- Båt i lä!\\
- Båt i lä!\\
- Det är far, det är vi!\\
- Det är far min från Smögen. Hallå!\\
- Båt i lä! sjöng han ut\\
- Dom är här jumpa i, alle man vi blir bärgade då\\
\end{vers}
\begin{vers}
Det var Stranne den äldre\\
En viking, en örn\\
Tog sitt renade brännvin\\
Ur vinskåpets hörn\\
Till att bjuda dom skeppsbrutna på\\
- Hur var namnet på skutan?\\
Han sporde och slog\\
Nio supar i spetsiga glas\\
- Briggen Blue bird\\
Det tionde glaset han tog\\
Och han slog det mot golvet i kras\\
- Sa ni Blue Bird kapten? Briggen Blue Bird av Hull?\\
- Gud i himlen var är då min son?\\
- Var är pojken kapten för vår frälsares skull?\\
Det blev dödstyst bland männen i vrån!\\
\end{vers}
\begin{vers}
Gubben Stranne\\
Tog sakta sydvästen utav\\
- Spara modern kapten, denna kväll\\
- Nämn ej namnet på briggen som har gått i kvav\\
- Nämn ej Blue Bird av Hull är ni snäll\\
Och kaptenen steg opp\\
Han var grå han var tärd\\
Stormen tjöt knappt man hörde hans ord\\
När han sa med självande röst till sin värd\\
- Karl stod surrad och glömdes ombord\\
\end{vers}
\end{song}
