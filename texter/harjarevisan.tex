\begin{song}{Härjarevisan}{harjarevisan}
\mel{Gärdebylåten\\ur Lundaspexet Djingis Kahn}
\begin{vers}
Liksom våra fäder \\
Vikingarna i Norden\\
Drar vi riket runt\\
Och super oss under borden\\
Brännvinet har blitt ett elixir\\
För kropp såväl som själ\\
Känner du dig liten\\
Och ynklig på jorden\\
Växer du med supen\\
Och blir stor uti orden\\
Slår dig för ditt håriga bröst\\
Och blir en man från hår till häl\\
\end{vers}
\begin{vers}
Ja, nu skall vi ut och härja: \\
Supa och slåss och svärja\\
Bränna röda stugor, slå små\\
Barn och säga fula ord\\
Med blod ska vi stäppen färga\\
Nu änteligen lär jag\\
Kunna dra nån riktigt nytta av min\\
Hermodskurs i mord\\
\end{vers}
\newp
\begin{vers}
Hurra nu ska man äntligen \\
Få röra på benen\\
Hela stammen jublar\\
Och det spritter i grenen\\
Tänk att än en gång\\
Få spränga fram på Brunte i galopp\\
Din doft o kära Brunte\\
Är trots brist i hygienen\\
För en vild mongol\\
Minst lika ljuv som syrenen\\
Tänk att på din rygg\\
Få rida runt i stan och spela topp!\\
\end{vers}
\begin{vers}
Ja nu ska vi ut och härja... \\
\end{vers}
\begin{vers}
Ja mordbränder är klämmiga, \\
Ta fram fotogenen\\
Och eftersläckningen\\
Tillhör just de fenomenen\\
Inom brandmansyrket som jag\\
Tycker är nån nytta med\\
Jag målar för mitt inre upp\\
Den härliga scenen;\\
Blodrött mitt i brandgult\\
Ej ens prins Eugen en\\
Lika mustig vy kan måla\\
Ens om han målade med sked\\
\end{vers}
\begin{vers}
Ja nu ska vi ut och härja..\\
\end{vers}
\end{song}
