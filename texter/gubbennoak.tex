\begin{song}{Gubben Noak}{fredmanssangno35}
\kom{Fredmans sång n:o 35}
\begin{vers}
Gubben Noak, gubben Noak\\
Var en hedersman\\
När han gick ur arken\\
Planterade han på marken\\
Mycket vin, ja mycket vin, ja\\
Detta gjorde han\\
\end{vers}
\begin{vers}
Noak rodde, Noak rodde\\
Ur sin gamla ark\\
Köpte sig buteljer\\
Sådana man säljer\\
För att dricka, för att dricka\\
På vår nya park\\
\end{vers}
\begin{vers}
Han väl visste, han väl visste\\
Att en mänska var\\
Torstig av naturen\\
Som de andra djuren\\
Därför han ock, därför han ock\\
Vin planterat har\\
\end{vers}
\begin{vers}
Gumman Noak, gumman Noak\\
Var en hedersfru\\
Hon gav man sin dricka\\
- fick jag sådan flicka\\
Gifte jag mig, gifte jag mig\\
Just på stunden nu\\
\end{vers}
\newp
\begin{vers}
Aldrig sad' hon, aldrig sad' hon\\
Kära far nå nå,\\
Sätt ifrån dig kruset;\\
Nej det ena ruset\\
På det andra, på det andra\\
Lät hon gubben få.\\    
\end{vers}
\begin{vers}
Gubben Noak, gubben Noak\\
Brukte egna hår,\\
Pipskägg, hakan trinder\\
Rosenröda kinder,\\
Drack i botten, drack i botten.\\
Hurra och gutår!\\
\end{vers}
\begin{vers}
Då var lustigt, då var lustigt\\
På vår gröna jord;\\
Man fick väl till bästa,\\
Ingen torstig nästa\\
Satt och blängde, satt och blängde\\
Vid ett dukat bord.\\
\end{vers}
\begin{vers}
Inga skålar, inga skålar\\
Gjorde då besvär,\\
Då var ej den läran:\\
Jag skall ha den äran;\\
Nej i botten, nej i botten\\
Drack man ur så här.\\
\end{vers}
\end{song}
