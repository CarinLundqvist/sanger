\begin{song}{Balladen om den kaxiga myran}{balladkaxigamyran}
\begin{vers}
Jag uppstämma vill min lyra\\
Fast det blott är en gitarr\\
Och berätta om en myra\\
Som gick ut att leta barr\\
Han gick ut i morgondiset\\
Sen han druckit sin choklad\\
Och försvann i lingonriset\\
\repopen Både mätt och nöjd och glad \repclose\\
\end{vers}
\begin{vers}
Det var långan väg att vandra\\
Det var långt till närmsta tall\\
Han kom bort ifrån dom andra\\
Men var glad i alla fall\\
Femti meter ifrån stacken\\
Just när solnedgången kom\\
Hitta' han ett barr på marken\\
\repopen Som han tyckte mycket om \repclose\\
\end{vers}
\begin{vers}
För att lyfta fick han stånka\\
Han fick spänna varje lem\\
Men så började han kånka\\
På det fina barret hem\\
När han gått i fyra timmar\\
Kom han till en ölbutelj\\
Han såg allting som i dimma\\
\repopen Bröstet hävdes som en bälg \repclose\\
\end{vers}
\begin{vers}
Den låg kvar sen förra lördan\\
- Jag skall släcka törsten min\\
Tänkte han och lade bördan\\
Utanför och klättra' in\\
Han drack upp den sista droppen\\
Som fanns kvar i den butelj\\
Och sedan slog han sig för kroppen\\
\repopen Och skrek ut: Jag är en älg! \repclose\\
\end{vers}
\begin{vers}
- Ej ett barr jag drar till tjället\\
Nu så ska jag tamejfan\\
Lämna skogen och i stället\\
Vända upp och ner på stan\\
Men han kom aldrig till staden\\
Något spärrade han stig\\
En koloss där låg bland bladen\\
\repopen Och vår myra hejdar sig \repclose\\
\end{vers}
\begin{vers}
Den var hiskelig att skåda\\
Den var stor och den var grå\\
Och vår myra skrek: - Anåda\\
Om du hindrar mig att gå!\\
Han for ilsken på kolossen\\
Som låg utsträckt i hans väg\\
Men vår myra kom ej loss sen\\
\repopen Han satt fast som i en deg \repclose\\
\end{vers}

\newp

\begin{vers}
Sorgligt slutar denna sången\\
Myran stretade och drog\\
Men kolossen höll'en fången\\
Tills han svalt ihjäl och dog\\
Undvik alkoholens yra:\\
Du blir stursk, men kroppen loj\\
Och om Du är född till en myra\\
\repopen - brottas aldrig med ett TOY \repclose\\
\end{vers}
\end{song}
