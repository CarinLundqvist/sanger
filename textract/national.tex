\begin{song}{Du gamla du fria}{dugamla}
\begin{vers}
Du gamla, du fria, du fjällhöga Nord,\\
du tysta, du glädjerika sköna.\\
Jag hälsar dig, vänaste land uppå jord,\\
din sol, din himmel, dina ängder gröna,\\
din sol, din himmel, dina ängder gröna.\\
\end{vers}
\begin{vers}
Du tronar på minnen från fornstora dar,\\
då ärat ditt namn flög över jorden.\\
Jag vet att du är och du blir, vad du var.\\
Ja, jag vill leva, jag vill dö i Norden,\\
ja, jag vill leva, jag vill dö i Norden.\\
\end{vers}
\begin{vers}
Jag städs vill dig tjäna mitt älskade land,\\
din trohet till döden vill jag svära.\\
Din rätt, skall jag värna, med håg och med hand,\\
din fana, högt den bragderika bära,\\
din fana, högt den bragderika bära.\\
\end{vers}
\begin{vers}
Med Gud skall jag kämpa, för hem och för härd,\\ 
för Sverige, den kära fosterjorden.\\ 
Jag byter Dig ej, mot allt i en värld\\ 
Nej, jag vill leva, jag vill dö i Norden,\\
nej, jag vill leva, jag vill dö i Norden.\\
\end{vers}
\end{song}

\newpage

\begin{song}{Ja, vi elsker}{vielsker}
\kom{Norges nationalsång}
\begin{vers}
Ja, vi elsker dette landet som det stiger frem,\\
Furet, værbitt over vannet med de tusen hjem.\\
Elsker, elsker vi og tenker på vår far og mor,\\
Og den saga natt som senker drømmer på vår jord,\\
Og den saga natt som senker, senker drømmer på vår jord.\\
\end{vers}
\begin{vers}
Norske mann i hus og hytte, takk din store Gud!\\
Landet ville han beskytte, skjønt det mørkt så ut.\\
Alt hva fedrene har kjempet, mødrene har grett,\\
Har den Herre stille lempet, så vi vant vår rett,\\
Har den Herre stille lempet, så vi vant, vi vant vår rett.\\
\end{vers}
\begin{vers}
Ja, vi elsker dette landet som det stiger frem,\\
Furet, værbitt over vannet med de tusen hjem.\\
Og som fedres kamp har hevet det av nød til seir,\\
Også vi, når det blir krevet, for dets fred slår leir!\\
Også vi, når det blir krevet, for dets fred, dets fred slår \\
\end{vers}
\end{song}

\newpage

\begin{song}{Vårt land}{vartland}
\kom{Finlands nationalsång}
\begin{vers}
Vårt land, vårt land, vårt fosterland,\\
ljud högt, o dyra ord!\\
Ej lyfts en höjd mot himlens rand,\\
ej sänks en dal, ej sköljs en strand,\\
mer älskad än vår bygd i nord,\\
än våra fäders jord.\\
\end{vers}
\begin{vers}
Din blomning sluten än i knopp,\\
skall mogna ur sitt tvång;\\
se, ur vår kärlek skall gå opp\\
ditt ljus, din glans, din fröjd, ditt hopp,\\
och högre klinga skall en gång,\\
vår fosterländska sång.\\
\end{vers}
\end{song}

\begin{song}{Det er et yndigt land}{yndigtland}
\kom{Danmarks nationalsång}
\begin{vers}
Der er et yndigt land,\\
det står med brede bøge\\
nær salten østerstrand;\\
det bugter sig i bakke, dal,\\
det hedder gamle Danmark,\\
og det er Frejas sal.\\
\end{vers}
\end{song}

\newpage

\begin{song}{Kungssången}{kungssangen}
\begin{vers}
Ur svenska hjärtans djup en gång,\\
en samfälld och en enkel sång\\
som går till kungen fram!   \\
Var honom trofast och hans ätt,\\
gör kronan på hans hjässa lätt,\\
och all din tro till honom sätt\\
du folk av frejdad stam!\\
\end{vers}
\begin{vers}
Du himlens Herre med oss var,\\
som förr Du med oss varit har,\\
och liva på vår strand.\\
Det gamla lynnets art igen\\
hos Sveakungen och hans män,\\
och låt din ande vila än\\
utöver nordanland!\\
\end{vers}
\end{song}

\newpage

\begin{song}{Ack Värmeland du sköna}{varmeland} 
\av{A. Fryxell}
\begin{vers}
Ack Värmeland, du sköna, du härliga land\\
Du krona bland Svea rikes länder!\\
Och komme jag än mitt i det förlovade land\\
Till Värmland jag ändock återvänder\\
Ja, där vill jag leva, ja, där vill jag dö\\
Om en gång ifrån Värmland jag tager mig en mö\\
Så vet jag, att aldrig jag mig ångrar\\ 
\end{vers}
\begin{vers}
I Värmeland är lustigt att leva och att bo\\
Det landet jag prisar så gärna\\
Där klappar det hjärtan med heder och med tro\\
Så fasta som bergenas kärna\\
Och var och en svensk uti Svea rikes land\\
Som kommer att gästa vid Klarälvens strand\\
han finner blott bröder och systrar\\
\end{vers}
\begin{vers}
I Värmeland - ja, där vill jag bygga och bo\\
Med enklaste lycka förnöjder\\
Dess dalar och skog ge mig tystnadens ro\\
Och luften är frisk på dess höjder\\
Och forsarna sjunga sin ljuvliga sång\\
Vid den vill jag somna så stilla en gång\\
Och vila i värmländska jorden\\ 
\end{vers}
\end{song}

\newpage

\begin{song}{Livet är härligt}{livet}
\mel{Polyushko polye, L. Knipper\\Ur Chalmersspexet Katarina II, 1959}
\begin{vers}
//: Livet är härligt!\\
Tavaritj, vårt liv är härligt!\\
Vi alla våra små bekymmer glömmer\\
när vi har fått en tår på tand, en SKÅL!\\
\end{vers}
\begin{vers}
Tag dig en vodka!\\
Tavaritj, en liten vodka!\\
Glasen i botten vi tillsammans tömmer,\\
det kommer mera efter hand ://\\
HEJ!\\
\end{vers}
\end{song}

\newpage

\begin{song}{Studentsången}{studentsangen}
\kom{Text: Herman Sätherberg, 1851\\Musik: Prins Gustaf}

\begin{vers}
Sjung om studentens lyckliga dag\\
låtom oss fröjdas i ungdomens vår!\\
Än klappar hjärtat med friska slag\\
och den ljusnande framtid är vår.\\
Inga stormar än i våra sinnen bo\\
hoppet är vår vän, och vi dess löften tro,\\
när vi knyta förbund i den lund\\
där de härliga lagrarna gro,\\
där de härliga lagrarna gro,\\
Hurra!\\
\end{vers}
\end{song}

\begin{song}{Rektorssången}{rektorssangen}
\mel{Kungssången}
\begin{vers}
Ur Chalmers hjärtans djup en gång,\\
en felstämd och en krånglig sång\\
som går till rektorn fram!   \\
Låt henne göra tentan lätt,\\
sätt mössan på'nas hjässa snett,\\
låt bäsken flöda vitt och brett\\
till törstig teknolog!\\
\end{vers}
\end{song}
