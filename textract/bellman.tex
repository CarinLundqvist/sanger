\begin{flushleft}
{\Huge Bellmanvisor\\}
\vspace{1cm}
\Large {Carl Michael Bellman (1740-1795) hör till de största svenska
skalderna. Han levde ett glatt liv i Stockholm och sågs ofta
på societetskrogen Den Gyllene Freden, liksom på sjaskiga hak på
Södermalm.\\
I dessa miljöer fann Bellman sin inspiration till visornas mustiga karaktärer - Movitz, Fredman, Ulla Winblad...

Vi gillar Bellman och har därför givit hans visor ett eget kapitel i
Encyclopedia Gasquica.}
\end{flushleft}
%BILD

\newpage

\begin{song}{Gubben Noak}{fredmanssangno35}
\kom{Fredmans sång n:o 35}
\begin{vers}
Gubben Noak, gubben Noak \\
var en hedersman.\\
När han gick ur arken,\\
planterade han på marken\\
mycket vin, ja mycket vin, ja\\
detta gjorde han.\\
\end{vers}
\begin{vers}
Noak rodde, Noak rodde \\
ur sin gamla ark.\\
Köpte sig buteljer,\\
sådana man säljer,\\
för att dricka, för att dricka\\
på vår nya park.\\
\end{vers}


\begin{vers}
Han väl visste, han väl visste \\
att en mänska var\\
torstig av naturen\\
som de andra djuren,\\
därför han ock, därför han ock\\
vin planterat har.\\
\end{vers}

\newp

\begin{vers}
Gumman Noak, gumman Noak \\
var en hedersfru.\\
Hon gav man sin dricka\\
- fick jag sådan flicka,\\
gifte jag mig, gifte jag mig\\
just på stunden nu.\\
\end{vers}
\end{song}


\begin{song}{Fjäriln vingad}{fjarilnvingad}
\kom{Fredmans sång n:o 64\\Dediceras til Herr Capitainen KJERSTEIN.}
\begin{vers}
Fjäriln vingad syns på Haga,\\
mellan dimmors frost och dun,\\
sig sitt gröna skjul tillaga\\
och i blomman, sin paulun.\\
Minsta kräk i kärr och syra,\\
nyss af Solens värma väckt,\\
til en ny högtidlig yra\\
eldas vid Zephirens flägt.\\
\end{vers}
\begin{vers}   
Haga, i ditt sköte röjes\\
gräsets brodd och gula plan.\\
Stolt i dina ränlar höjes\\
gungande den hvita Svan.\\
Längst ur skogens glesa kamrar\\
höras täta återskall,\\
än från den graniten hamrar,\\
än från yx i björk och tall.\\
\end{vers}
\end{song}


\begin{song}{Så lunka vi}{fredmanssangno21}
\kom{Fredmans sång n:o 21}
\begin{vers}
Så lunka vi så småningom \\
från Bacchi buller och tumult,\\
när döden ropar - Granne kom,\\
ditt timglas är nu fullt.\\
Du gubbe fäll din krycka ner,\\
och du du yngling, lyd min lag,\\
den skönsta nymf som mot dig ler\\
inunder armen tag.\\
Tycker du att graven är för djup,\\
nå välan, så tag dig då en sup,\\
tag dig sen dito en, dito två, dito tre,\\
så dör du nöjdare.\\
\end{vers}
\begin{vers}
Men du som med en trumpen min, \\
bland riglar, galler, järn och lås,\\
dig vilar på ditt penningskrin,\\
inom din stängda bås.\\
Och du som svartsjuk slår i kras\\
buteljer, speglar och pokal;\\
bjud nu god natt, drick ut ditt glas,\\
och hälsa din rival.\\
Tycker du ....\\
\end{vers}
\newp
\begin{vers}
Men du som med en ärlig min\\
plär dina vänner häda jämt,\\
och dem förtalar vid ditt vin\\
och det liksom på skämt.\\
Och du som ej försvarar dem\\
fastän ur deras flaskor du,\\
du väl kan slicka din fem,\\
vad svarar du väl nu?\\
-Tycker du att...\\
\end{vers}
\begin{vers}
Säg är du nöjd, min granne säg, \\
så prisa värden nu till slut,\\
om vi ha en och samma väg,\\
så följoms åt - drick ut.\\
Men först med vinet rött och vitt\\
för vår värdinna bugom oss,\\
och halkom sen i graven fritt,\\
vid aftonstjärnans bloss.\\
Tycker du ...\\
\end{vers}
\end{song}


\newpage


\begin{song}{Märk hur vår skugga}{epistel81}
\kom{Fredmans epistel n:o 81\\Til Grälmakar Löfberg i Sterbhuset vid\\ 
Dantobommen, diktad vid Grafven.\\Dedicerad til Doctor BLAD.}
\begin{vers}           
Märk hur' vår skugga, märk Movitz Mon Frere!\\
Inom et mörker sig slutar,\\
Hur Guld och Purpur i Skåfveln, den där,\\
Byts til grus och klutar.\\
Vinkar Charon från sin brusande älf,\\
Och tre gånger sen Dödgräfvaren sjelf,\\
Mer du din drufva ej kryster.\\
Därföre Movitz kom hjelp mig och hvälf\\
Grafsten öfver vår Syster.\\
\end{vers}
\begin{vers}          
Ach längtansvärda och bortskymda skjul,\\
Under de susande grenar,\\
Där Tid och Döden en skönhet och ful\\
Til et stoft förenar!\\
Til dig aldrig Afund sökt någon stig,\\
Lyckan, eljest uti flygten så vig,\\
Aldrig kring Grifterna ilar.\\
Ovän där väpnad, hvad synes väl dig?\\
Bryter fromt sina pilar.\\
\end{vers}

\newp

\begin{vers}
Lillklockan klämtar til Storklockans dön,\\
Löfvad står Cantorn i porten;\\
Och vid de skrålande Gåssarnas bön,\\
Helgar denna orten.\\
Vägen opp til Templets griftprydda stad\\
Trampas mellan Rosors gulnade blad,\\
Multnade Plankor och Bårar;\\
Til dess den långa och svartklädda rad,\\
Djupt sig bugar med tårar.\\
\end{vers}
\begin{vers}
Så gick til hvila, från Slagsmål och Bal,\\
Grälmakar Löfberg, din maka; \\
Där, dit åt gräset, långhalsig och smal,\\
Du än glor tilbaka.\\
Hon från Dantobommen skildes i dag,\\
Och med Hänne alla lustiga lag;\\ 
Hvem skall nu Flaskan befalla.\\
Torstig var hon och uttorstig är jag;\\ 
Vi är torstiga alla.\\
\end{vers}
\end{song}

\newpage


\begin{song}{Gutår båd natt och dag}{epistel01}
\kom{Fredmans epistel n:o 1}
\begin{vers}
Gutår båd natt och dag!\\
Ny vällust, nytt behag!\\
Fukta din aska;\\
Fram brännvins flaska;\\
Lydom Bacchi lag;\\
Gutår båd natt och dag!\\
Si vår Syster Cajsa Stina;\\
Si hur hännes flaskor skina.\\
Kära ta hit stopet, kära tag hit stopet, grina;\\
Grina, svälj och drick, som jag.\\
\end{vers}
\begin{vers}
Gutår, ett laga fång;\\
Vår sorgedag är lång;\\
Lång är Buteljen;\\
Trumla Reveljen;\\
Supom om en gång;\\
Vår sorgedag är lång.\\
Cajsa Stina står och tappar;\\
Hela hjertat i mig klappar;\\
Bara ingen stopet, bara ingen stopet nappar;\\
Då gör jag min svane-sång.\\
\end{vers}
\end{song}


\newpage

\begin{song}{Solen glimmar blank och trind}{solenglimmar}
\kom{Fredmans epistel n:o 48,\\Hvaruti afmålas Ulla 
Winblads hemresa från\\Hessingen i Mälaren en sommar-morgon 1769.\\}

\begin{vers}
Solen glimmar blank och trind,\\
Vattnet likt en spegel;\\
Småningom upblåser vind\\
I de fallna segel;\\
Vimpeln sträcks, och med en år\\
Olle på en Höbåt står;\\
Kerstin ur Kajutan går,\\
Skjuter lås och regel.\\
\end{vers}
\begin{vers}
Norström stjelper sin peruk\\
Af sin röda skalle,\\
Och min Ulla blek och sjuk\\
Lät sin kjortel falla,\\
Klef så bredbent i paulun;\\
Movitz efter med basun:\\
Maka åt dig Norström! Frun\\
Hör ju til oss alla.\\
\end{vers}
\end{song}
\kom{(Då orginalsången har 21 verser valde\\
  vi att ta med den första och sista)}

%Liksom en herdinna

