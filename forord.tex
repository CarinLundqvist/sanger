\begin{flushleft}
{\Huge Förord\\}
\end{flushleft}

{\large
I handen har du nu Encyclopedia Gasquica, den sjungbok som finns för medlemmar av Fysikteknologsektionen.
Boken är en del i den långa tradition som funnits på Chalmers av att anordna sittningar.

Även om sittningarna i tidernas begynnelse såg annorlunda ut, så har de mycket gemensamt med idag.
En god måltid inmundigas, det sjungs och det dricks.
Kom dock ihåg att god sed kännetecknas av att inte dricka utan att sjunga.
Ty, kan man inte sjunga längre är det en god indikation på att man skall minska intagandet av dryck.

Jag som i mitt anletes svett har editerat detta magnifika verk, heter Thomas.
Sjungboken hade dock aldrig funnits utan alla tidigare sångförmän, diktare, låtskrivare, Bellman, Jonas Ahlströmer, Chalmersspexet, Barocken,
Dennis, Elmer, Preisz, Kvickan, Roine, Ewok, An'li, Anders Eldeman, Olofponken, CING, många fler eller en god whisky.
Jag och alla som har hjälpt mig vill värna om en sångskatt som decennier av fysikteknologer har förädlat.
Vem kan glömma gamla fysikaliska klassiker som ''Man ska ha MATLAB'', ''Min punsch'' eller ''Bara kvant''?

Det är med stolthet och framtidstro som jag till sist vill önska dig stor glädje vid nyttjandet av sjungboken.

\vspace{0.5cm}
\begin{flushright}
\textbf{Thomas Lovén}
Sångförman 2007-2010
\end{flushright}
}

\newpage

{\large
Med nya tider kommer nya teknologier.
Med nya teknologier kommer nya möjligheter.

\vspace{0.5cm}
\begin{flushright}
\textbf{Johan Winther}
Sångförman 2015-2018
PR-chef i F6 15/16
Vice ordförande i SNF 16/17
Informationsansvarig i F-styret 17/18
\end{flushright}
}

\newpage

\begin{flushleft}
{\Huge Dryckesregler\\}
\end{flushleft}
{\large
'den som sviker uti dryckjom sviker ock i androm stjyckjom'
\begin{itemize}
\item Tersen har tagits. Quarten bjudes.\\ Dryckesbroder säger: 'Tack, jag 
skall nog inte ha mer'.\\ Det  är barnjoller, böte 2 öre.
\item Någon dricker ur annans glas. \\Det är nidingsverk, böte 10 öre.
\item Någon kallar en annan i två dryckesbröders närvaro 'nykterist'. 
Det är okvädinsord, böte 15 öre.
\item Dryckesbroder sjunger samma nubbevisa 3 gånger å rad. Han vare var 
mans niding.
\item Spiller någon ur eget glas, det är slarv. Han skylle sig själv. 
Spiller någon ur flaskan, det är nidingsverk, böte 5 öre.
\item Någon säger till en annan efter tersen: 'Jag tål mer än du'. Det 
är okvädinsord. De skola mötas nästa dag vid bord och bägare. De 
dricka, tills en av dem faller under bordet. Blir den liggande, som ord 
gav, han skall ligge ogill.
\end{itemize}

\begin{flushleft}
{\Huge Avtackning\\}
Som tack för ett framträdande sjungs följande:

Det där det gjorde de fan så bra!\\
En skål uti botten för NN vi ta\\
\repopen Hugg i och dra, hej! \repclose\\
En skål uti botten för NN vi ta\\
Och alla så dricka vi nu NN till\\
Och NN säger\\
(Solo NN)\\
För det var i vår ungdoms fagraste vår\\
Vi drack varandra till och vi sade gutår!\\
\end{flushleft}
}

































