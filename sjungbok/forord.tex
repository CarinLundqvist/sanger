\begin{flushleft}
\section{Förord}
\end{flushleft}

{\large
\setlength{\parskip}{0.8em}
I handen har du nu Encyclopedia Gasquica, den sjungbok som finns för medlemmar av Fysikteknologsektionen.
Boken är en del i den långa tradition som funnits på Chalmers av att anordna sittningar.

Även om sittningarna i tidernas begynnelse såg annorlunda ut, så har de mycket gemensamt med idag.
En god måltid inmundigas, det sjungs och det dricks.
Kom dock ihåg att god sed kännetecknas av att inte dricka utan att sjunga.
Ty, kan man inte sjunga längre är det en god indikation på att man skall minska intagandet av dryck.

Jag som i mitt anletes svett har editerat detta magnifika verk, heter Thomas.
Sjungboken hade dock aldrig funnits utan alla \\tidigare sångförmän, diktare, låtskrivare, musiker, Bellman, Jonas Ahlströmer, Chalmersspexet, Barocken,
Dennis, Elmer, Preisz, Kvickan, Roine, Ewok, An'li, Anders Eldeman, Olofponken, CING, många fler eller en god whisky.
Jag och alla som har hjälpt mig vill värna om en sångskatt som decennier av fysikteknologer har förädlat.
Vem kan glömma gamla fysikaliska klassiker som ''Man ska ha MATLAB'', ''Min punsch'' eller ''Bara kvant''?

Det är med stolthet och framtidstro som jag till sist vill önska dig stor glädje vid nyttjandet av sjungboken.

\vspace{0.5cm}
\begin{flushright}
\textbf{Thomas Lovén}\\
Sångförman 2007-2010
\end{flushright}
}

\newpage

\begin{flushleft}
\section{Förord}
\end{flushleft}

{\large
\setlength{\parskip}{0.8em}
\textit{Med nya tider kommer nya teknologier.}\\
\textit{Med nya teknologier kommer nya möjligheter.}

När sjungbokens frö såddes fanns det en vision om en ljuvstämmig framtid där alla sektionens medlemmar kan förvalta och bevara sektionens sångtraditioner tillsammans. 
Denna tradition sträcker sig ytterst långt tillbaka i studenternas historia och nu är det dags för dig att bära den vidare.

Fröet har växt och är nu redo att skördas i formen av den sjungbok du håller i nu.
Detta är en samling av de vanligaste sångerna på sektionen plus flera andra som du kan råka höra eller sjunga under din tid som teknolog och kan därför visa sig nyttig även utanför sektionens trygga mörker.

Bär med dig denna bok och låt sångernas melodier fylla sittningar, kalas och andra evenemang så kan vi tillsammans göra den enda tystnaden till den mellan pausen och verserna.

\vspace{.5cm}
\begin{flushright}
\textbf{Albert Vesterlund}\\
Sångförman 2021\\
\end{flushright}
}

%{\large
%\setlength{\parskip}{0.8em}
%\textit{Med nya tider kommer nya teknologier.}\\
%\textit{Med nya teknologier kommer nya möjligheter.}
%
%Min resa i sångernas värld började år 2014 när jag blev Nollan på Teknisk fysik. Jag insåg efter en rad sittningar att sånghäftena inte var särskilt tilltalande. Det visade sig också att det inte heller fanns ett snabbt, enkelt och standardiserat sätt att skapa dem.
%Därför började jag under aspning för F6 2015 att utveckla en LaTeX-mall för sånghäften i hopp om att kunna använda den under mitt år som sexmästerist.
%Det blev en stor succé och medlemmar från alla sektionens hörn använde min mall.

%Jag insåg att det fanns mer att göra och bestämde mig därför att åta uppdraget som sångförman 2016.
%Jag hade en vision om ett sånghäfte och en sjungbok som levde i systemisk symbios.
%Först lade jag upp den dåvarande sjungboken på sektionens hemsida och därmed kom sångförmännen in i millennium två.
%Sånghäftesmallen blev ockå tillgänglig på hemsidan, så att den inte skulle falla i glömska.
%Men det tog mig ett år att samla modet för att uppnå min vision, då jag i början av sommaren 2017 släppte sjungbokens och sånghäftets källkod på sektionens GitHub.
%Efter dagars kodande, debuggande och frenetiskt letande på StackExchange var det klart.
%
%Jag har därmed, via sångförmännen, grott det första fröet för en ljuvstämmig framtid; en framtid där \textbf{alla} sektionens medlemmar kan förvalta och bevara sektionens sångtraditioner tillsammans. Och jag hoppas att du vill vara med på resan...
%
%\begin{flushright}
%\textbf{Johan ''Wello'' Winther}\\
%Sångförman 2016-2018\\
%PR-chef i F6 15/16\\
%Vice ordförande i SNF 16/17\\
%Informationsansvarig i F-styret 17/18\\
%\end{flushright}
%}

\newpage
\begin{flushleft}
\section{Supregler anno 1900}
\end{flushleft}
{\large
Den ganska märkliga svenska supseden är en fullkomlig nationell egenart, som inte tillämpas utomlands.
Dess historia kan spåras till urtiden, varifrån den konserverats inom kloster och sedemera fortlevat inom skråväsendet.
Vikingarna saknade överflöd på dryckeskärl, varför ett och samma horn ofta gick från man till man runt bänklaget.
Först drack hövdingen.
Därmed blev drickandet fritt i det att övriga deltagare i kalaset fingo tillfälle att begagna kärlet.
När detta gått laget runt och återkommmit till hövdingen var drickandet slut.
Tal hålles inte till brännvin.
Orsaken därtill kan sökas i det svenska brännvinsglasets konstruktion.
Detta föreställer en tratt och var även till sitt ursprung en sådan.
För att hålla drycken kvar, måste man täppa pipen med ett finger.
På det sättet kunde man inte i oändlighet hålla brännvinet i beredskap
utan tvingades att hastigt svälja supen i ett svep för att få fingret och handen fria.
Med utgångspunkt från dessa traditionella förhållanden
är det lättare att förstå de numera gällande lagarna för svenskt skålande.
\vspace{.8em}
\begin{enumerate}
\item En skål är ett tyst tal för den som tilldrickas.
\item En skål får ej föreslås innan det första talet hållits.
\item Det åligger värden att dricka med varje gäst.
\item Den som fått emottaga en skål är skyldig att besvara den.
\item Personer av lägre rang bör ej dricka med person av högre rang,
innan denne druckit med den lägre placerad.*
\end{enumerate}
\newpage

* Rangrulla:
\begin{itemize}
    \item Egen operaloge ger: 6 poäng.
    \item Eget stenhus i staden ger: 5 poäng.
    \item Eget hus på landet ger: 4 poäng.
    \item Varje miljon kronor på banken ger: 2 poäng.
    \item Egen landå med fyrspann ger: 1 poäng.
    \item En vinkällare ger: 1 poäng.
    \item En känd anhörig ger $\frac{1}{2}$ poäng.
    \item God allmänbildning ger $\frac{1}{8}$ poäng.
\end{itemize}
}

\begin{flushleft}
\section{Avtackning}
\end{flushleft}
{\large
Som tack för ett framträdande sjunges följande:

\begin{flushleft}
Det där det gjorde de fan så bra!\\
En skål uti botten för NN vi ta\\
\repopen Hugg i och dra, hej! \repclose\\
En skål uti botten för NN vi ta\\
Och alla så dricka vi nu NN till\\
Och NN säger\\
(Solo NN)\\
För det var i vår ungdoms fagraste vår\\
Vi drack varandra till och vi sade gutår!\\
\end{flushleft}
}



